\documentclass{article}

\title{Software Engineering - Lecture 3}
\author{Ossama Edbali}

\begin{document}
	
	\maketitle	
	
	\section{Testing}	
	Testing is a very important step in a software process. It allows to:
	\begin{itemize}
		\item Prevent bugs before the production and maintenance step
		\item Check the functionality of the application whether it is working as per requirements
		\item Making the overall system reliable
	\end{itemize}
	
	Some useful definitions:
	\begin{description}
		\item[SUT] System Under Testing
		\item[DOC] Dependent On Component
	\end{description}		
	
	\subsection*{Levels of testing}
	There are three levels of testing each distinguished by the components to test and their dependencies.
	
	\subsubsection*{Unit testing}	
	The purpose here is to test single classes or methods in isolation.
	There are two types of testing:
	\begin{description}
		\item[Black box] Choose test data without looking at implementation. Test just I/O
		\item[Glass/white box] Choose test data with knowledge of implementation.
		Test the internal logic of the subsystem or object
	\end{description}
	
	\subsubsection*{Integration testing}
	Tests different parts (modules, classes) of the system working together. It is done, like unit testing,
	by the developer.	
	
	\subsubsection*{End-to-end testing}	
	Test the whole system including the external resources such as databases, APIs etc\ldots
	End-to-End testing requires no access to or understanding of the code base.
	It is carried out by an independent user which is driven by quality rather than delivery (like the
	developer in unit testing). In this stage the overall system is tested under stress in order to
	cover the maximum number of edge cases.
	
	\subsection*{Process}	
	The basic steps in testing are:
	\begin{itemize}
		\item Choose input data/configuration
		\item Define the expected outcome
		\item Run program/method against the input and record the results
		\item Examine results against the expected outcome
	\end{itemize}
	
	There are different approaches to testing: manual and automated. Manual testing is more tedious and
	error-prone whereas automated testing is more powerful using a test framework.
	
	We are going to use the TestNG (Next Generation) Open source automated testing framework inspired
	from JUnit but introduces additional features in order to deal with integration and end-to-end testing.
	(installation and usage in the lecture notes).
	
\end{document}