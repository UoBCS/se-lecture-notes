\documentclass{article}

\title{Software Engineering - Lecture 4}
\author{Ossama Edbali}

\begin{document}
	
	\maketitle	
	
	\section{Parametrised testing in TestNG}	
	It is often appropriate to test the same method multiple times with different parameters (e.g. prime
	number checking, celcius to fahrenheit converter etc\ldots).
	
	TestNG provides Parameterised tests to more conveniently handle such sets of tests.
	A Parameterised Test is a normal test method which has a mechanism associated with it for running the same
	test multiple times with different parameters. The simplest form in TestNG is to specify the test method
	with
	an annotation that identifies another method that provides the sets of parameters to use — a DataProvider
	method.
	
	Then the data source is a private (usually) method which is annotated with \textit{DataProvider} and
	returns an array of object arrays.
	
	\section{TDD - Test Driven Development}
	see module handouts.	
	
	\section{Mock objects}
	see module handouts	
	
\end{document}