\documentclass{article}

\title{Software Engineering - Lecture 14}
\author{Ossama Edbali}

\begin{document}
	
	\maketitle
	
	\section*{Introduction to agile methods}
	Agile methods became popular in the beginning of the 2000s. This is because the market
	was very competitive, therefore software teams rethought the software processes. Their aim was
	to deliver quality products within specific deadlines and with a reduced cost.	
	
	One might spend a lot of time in the requirements phase but after a period they may change or become unsuitable.
	
	\section*{Agile methodology}
	\begin{itemize}
		\item Incremental and iterative
		\item Frequent early release
		\item Streamlined
		\item Time-boxed
		\item Collaborative
	\end{itemize}		
	
	The main features of agile methods are:
	\begin{itemize}
		\item There is no need for a detailed specification of the system
		\item Design and documentation are minimised
		\item Specification, design and implementation are parallel processes
		\item System developed incrementally in versions or releases
		\item Customers are heavily involved in the specification and testing
		\item Increments are short (2 - 4 weeks)
		\item Documentation is minimised
	\end{itemize}
	
	\section*{Scrum}
	See lecture notes
		
\end{document}