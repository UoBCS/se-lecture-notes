\documentclass{article}

\title{Software engineering - Lecture 1}
\author{Ossama Edbali}

\begin{document}

	\maketitle
	
	\section{Module outline}
	The instructor of this module is Shereen Fouad:
	\begin{description}
		\item[Room] 227
		\item[Email] fouadsa@cs.bham.ac.uk
		\item[Item hours] Mondays (2pm - 3pm)
	\end{description}
	
	\subsection*{Grading}
	\begin{itemize}
		\item 20\% of continuous assessment (1 team exercises + 2 individual online tests)
		\item 80\% of examination mark
	\end{itemize}		
	
	\section{Introduction to SE}
	Software engineering deals with delivering high quality software within a
	limited amount of budget and time. But what is software?
	
	Software is a product that contains computer programs, libraries
	and their associated documentation such as requirements, design models and user manuals.
	The attributes of a good software are:
	\begin{itemize}
		\item high quality: maintainable, dependable and acceptable
		\item budget consideration
		\item respect of deadline
	\end{itemize}		
	
	Software engineering is an engineering discipline
	that is concerned with all aspects of software production. Software engineering
	techniques are vital and very useful when dealing with large scale problems.
	
	In software engineering we define four figures:
	\begin{itemize}
		\item \textbf{Customer}: Requires a computer system to achieve some business goals
		by user interaction or interaction with the environment
		in a specified manner
		\item \textbf{User} of the system
		\item \textbf{Software engineer}: To understand how the system-to-be
		needs to interact with the user or the environment so that
		customer's requirement is met and design the software-to-be
		\item \textbf{Programmer}: To implement the software-to-be
		designed by the software engineer
	\end{itemize}
	
	An important point is the difference between computer science and software engineering.
	CS is dealing with the theoretical aspects of computation whereas software engineering
	is concerned with the design of high quality software.
	
	\section{Contents of this module}
	\begin{itemize}
		\item Software Process Models
		\item Software Testing
		\item Software Design using UML class diagrams
		\item Structured Design
		\item Use cases
		\item Agile process models
		\item State Machine Diagrams
		\item Sequence Diagrams
		\item Refactoring
	\end{itemize}
	
\end{document}