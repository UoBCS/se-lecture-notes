\documentclass{article}

\title{Software Engineering - Lecture 15}
\author{Ossama Edbali}

\begin{document}
	
	\maketitle
	
	\section*{Introduction to extreme programming}
	The best known agile method is extreme programming. The characteristics of this process are:
	\begin{itemize}
		\item \textbf{Cyclic process} - each cycle takes 2 weeks approx.
		\item Planning and iterations are based on \textbf{user stories}
		\item \textbf{User stories} form the basis for system requirements.
		\item \textbf{Whole-team}: all contributors to an XP project are considered as one team
		\item \textbf{Small releases}: the team releases working software at each iteration. Each release
		adds functionality on top of the previous release.
		\item \textbf{Customer tests}: the customer defines one or more automated acceptance tests for a feature.
		\item \textbf{Simple design}
		\item \textbf{Refactoring}: see separate lecture
		\item \textbf{Pair programming}: cooperation leads to better understanding. Better initial approaches
		and less back-tracking.
		\item \textbf{TDD}: XP is an incremental development approach, we do not have the detailed requirements and
		specifications that we would expect to have in a plan-driven approach
	\end{itemize}
	
	\subsection*{Planning}
	There are 2 types of planning in software development:
	\begin{itemize}
		\item \textbf{Release plan}: predict what will be accomplished by the due date
		\item \textbf{Iteration plan}: determine what tasks to accomplish next
	\end{itemize}
	
	Iteration planning works as follows:
	\begin{itemize}
		\item Choose user stories for the current cycle and \textbf{failed acceptance tests} from the previous one 
		\item Break them into tasks
		\item Tasks are written on cards by developers
		\item Developers implement each task separately
		\item Test and integrate software
		\item Release software
		\item Acceptance testing
	\end{itemize}
	
	During the process of planning a task CRCs (Class Responsibility Collaborator) are used to determine
	how a class should be developed.
	
	\subsection*{Why refactoring in XP?}
	Refactoring is extremely important in XP because there are continuous changes to the code base: this may
	lead to badly written code. Therefore a solution is to refactor the code to make it cleaner, readable
	and maintainable.
		
\end{document}